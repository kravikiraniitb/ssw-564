This document provides the requirements and design details of the
PROJECT.  The following table (Table \ref{Table::UpdateHistory}) should be
updated by authors whenever major changes are made to the architecture
design or new components are added.  

\begin{longtable}{|l||p{13.5cm}|}
%\centering
\caption{Document Update History \label{Table::UpdateHistory}}\\
\hline
\textbf{Date} & \textbf{Updates} \\
\hline 
\endhead

11/10/2022 & Ravi Kiran:
\begin{itemize}[topsep=0pt,itemsep=0pt,parsep=0pt,partopsep=0pt,leftmargin=12pt]
\item Testing.
\item Added the documentation logic and figure for dmaLassoInterpolate 
(Chapter \ref{Chapter::dmaLassoInterpolate}). 
\end{itemize} 
\\ \hline

12/16/2021 & DDM:
\begin{itemize}[topsep=0pt,itemsep=0pt,parsep=0pt,partopsep=0pt,leftmargin=12pt]
\item Modified drawings of Figure \ref{Figure::dmaRemoveBackgroundSubclass} to
highlight the new mask creation subclass dmaRBMaskCreateMagicLasso (Chapter 
\ref{Chapter::dmaRBMaskCreateMagicLasso}).
\item Added two new chapters: dmaRBMaskCreateMagicLasso
(Chapter \ref{Chapter::dmaRBMaskCreateMagicLasso}) and dmaLassoInterpolate
(Chapter \ref{Chapter::dmaLassoInterpolate}) to capture the latest logic for 
background removal.  A user uses the lasso around the object, or the average 
of 360 objects, and the magic lasso mask creation will interpolate the missing
pixels and then use that as the background image. 
\end{itemize} 
\\ \hline

12/16/2021 & Initials:
\begin{itemize}[topsep=0pt,itemsep=0pt,parsep=0pt,partopsep=0pt,leftmargin=12pt]
\item Modified drawings of Figure \ref{Figure::dmaRemoveBackgroundSubclass} to
highlight the new mask creation subclass dmaRBMaskCreateMagicLasso (Chapter 
\ref{Chapter::dmaRBMaskCreateMagicLasso}).
\item Added two new chapters: dmaRBMaskCreateMagicLasso
(Chapter \ref{Chapter::dmaRBMaskCreateMagicLasso}) and dmaLassoInterpolate
(Chapter \ref{Chapter::dmaLassoInterpolate}) to capture the latest logic for 
background removal.  A user uses the lasso around the object, or the average 
of 360 objects, and the magic lasso mask creation will interpolate the missing
pixels and then use that as the background image. 
\end{itemize} 
\\ \hline

12/16/2021 & DDM:
\begin{itemize}[topsep=0pt,itemsep=0pt,parsep=0pt,partopsep=0pt,leftmargin=12pt]
\item Modified drawings of Figure \ref{Figure::dmaRemoveBackgroundSubclass} to
highlight the new mask creation subclass dmaRBMaskCreateMagicLasso (Chapter 
\ref{Chapter::dmaRBMaskCreateMagicLasso}).
\item Added two new chapters: dmaRBMaskCreateMagicLasso
(Chapter \ref{Chapter::dmaRBMaskCreateMagicLasso}) and dmaLassoInterpolate
(Chapter \ref{Chapter::dmaLassoInterpolate}) to capture the latest logic for 
background removal.  A user uses the lasso around the object, or the average 
of 360 objects, and the magic lasso mask creation will interpolate the missing
pixels and then use that as the background image. 
\end{itemize} 
\\ \hline

12/16/2021 & Initials:
\begin{itemize}[topsep=0pt,itemsep=0pt,parsep=0pt,partopsep=0pt,leftmargin=12pt]
\item Modified drawings of Figure \ref{Figure::dmaRemoveBackgroundSubclass} to
highlight the new mask creation subclass dmaRBMaskCreateMagicLasso (Chapter 
\ref{Chapter::dmaRBMaskCreateMagicLasso}).
\item Added two new chapters: dmaRBMaskCreateMagicLasso
(Chapter \ref{Chapter::dmaRBMaskCreateMagicLasso}) and dmaLassoInterpolate
(Chapter \ref{Chapter::dmaLassoInterpolate}) to capture the latest logic for 
background removal.  A user uses the lasso around the object, or the average 
of 360 objects, and the magic lasso mask creation will interpolate the missing
pixels and then use that as the background image. 
\end{itemize} 
\\ \hline


\end{longtable}


